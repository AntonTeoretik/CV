%%%%%%%%%%%%%%%%%%%%%%%%%%%%%%%%%%%%%%%%%
% Developer CV
% LaTeX Template
% Version 1.0 (28/1/19)
%
% This template originates from:
% http://www.LaTeXTemplates.com
%
% Authors:
% Jan Vorisek (jan@vorisek.me)
% Based on a template by Jan Küster (info@jankuester.com)
% Modified for LaTeX Templates by Vel (vel@LaTeXTemplates.com)
%
% License:
% The MIT License (see included LICENSE file)
%
%%%%%%%%%%%%%%%%%%%%%%%%%%%%%%%%%%%%%%%%%

%----------------------------------------------------------------------------------------
%	PACKAGES AND OTHER DOCUMENT CONFIGURATIONS
%----------------------------------------------------------------------------------------

\documentclass[9pt]{developercv} % Default font size, values from 8-12pt are recommended

%----------------------------------------------------------------------------------------

\begin{document}

%----------------------------------------------------------------------------------------
%	TITLE AND CONTACT INFORMATION
%----------------------------------------------------------------------------------------

\begin{minipage}[t]{0.45\textwidth} % 45% of the page width for name
	\vspace{-\baselineskip} % Required for vertically aligning minipages
	
	% If your name is very short, use just one of the lines below
	% If your name is very long, reduce the font size or make the minipage wider and reduce the others proportionately

	\Huge{ANTON \\ABLOV}
	
	\vspace{6pt}
	
\end{minipage}
\begin{minipage}[t]{0.275\textwidth} % 27.5% of the page width for the first row of icons
	\vspace{-\baselineskip} % Required for vertically aligning minipages
	
	% The first parameter is the FontAwesome icon name, the second is the box size and the third is the text
	% Other icons can be found by referring to fontawesome.pdf (supplied with the template) and using the word after \fa in the command for the icon you want
	Phone: +49 1515 994 83 17\\
	E-mail:  {\href{mailto:a.d.ablov@gmail.com}{a.d.ablov@gmail.com}}\\	
\end{minipage}
\begin{minipage}[t]{0.275\textwidth} % 27.5% of the page width for the second row of icons
	\vspace{-\baselineskip} % Required for vertically aligning minipages
	
	% The first parameter is the FontAwesome icon name, the second is the box size and the third is the text
	% Other icons can be found by referring to fontawesome.pdf (supplied with the template) and using the word after \fa in the command for the icon you want
	%\icon{Globe}{12}{\href{https://alyx.vance.me}{alyx.vance.me}}\\
	Github:\,\href{https://github.com/AntonTeoretik}{github.com/AntonTeoretik}
	LinkedIn:\,\href{https://www.linkedin.com/in/antonablov/}{linkedin.com/in/antonablov}
	
\end{minipage}

\vspace{0.5cm}

%----------------------------------------------------------------------------------------
%	INTRODUCTION, SKILLS AND TECHNOLOGIES
%----------------------------------------------------------------------------------------

\cvsect{Who Am I?}

\begin{minipage}[t]{0.6\textwidth} % 40% of the page width for the introduction text
	\vspace{-\baselineskip} % Required for vertically aligning minipages
	I am student, currently finishing my masters. My main interests are math and programming. So, my skills correspond to these topics -- I have an experience in various programming languages like C++, Haskell, Python, which I use to write some math (and other) projects, for example fluid modeling. I'm like to learn new things, especially if they are appliable and difficult simultaneously.
	
	
\end{minipage}
\hfill % Whitespace between
\begin{minipage}[t]{0.5\textwidth} % 50% of the page for the skills bar chart
	\vspace{-\baselineskip} % Required for vertically aligning minipages

\end{minipage}


%----------------------------------------------------------------------------------------
%	EXPERIENCE
%----------------------------------------------------------------------------------------

\cvsect{Experience}

\begin{entrylist}
	\entry
	{Jul-Oct 2020}
	{Haskell developer}
	{National center for continuous science education}
	{
		I was a back-end developer in project <<Hypermathica>> \href{https://7.math.ru/}{(7.math.ru)} -- an online educational platform for school children. 
	}

	\entry
	{Jul 2019 — \\Jun 2020}
	{Methodist}
	{National center for continuous science education}
	{
		I worked at content part for an online educational platform for school children \href{https://7.math.ru/}{(7.math.ru)}. Big part of my work was also writing scripts on Python, which help to automatize different tasks.
	}

	\entry
	{Jun-Aug 2018}
	{C++ developer (Internship) }
	{Fraunhofer Institute, Kaiserslautern}
	{
		I was part of a team that developed a program that simulated fiber structures for the needs of textile companies.
		The project was leaded by Dr Julia Orlik
		(\href{mailto:julia.orlik@itwm.fraunhofer.de}{julia.orlik@itwm.fraunhofer.de}).
	}
	
	\entry
		{2016 -- 2020}
		{Mathematics and programming teacher}
		{ 
		Delta Summer Camp, Munich  \\
		\null\hfill \href{https://delta.camp/}{delta.camp}	
		}
		{
		For five years in Delta I taught several mathematical courses, such as <<Introduction into the Group Theory>>, <<Topology>>, <<Stories about Infinite numbers>>, <<Paradoxes of Set Theory>> and	 etc. 		
		In addition, I made several projects with the students, written in various programming languages, for example:\\
		- maze generator, written in Python;\\
		- physical processes simulation (for instance 2D fluid flow ) written in Python and C++;\\
		- electromagnetic field visualization written in Haskell + OpenGl;
		}
	
	\entry
	{2015 -- 2019}
	{Mathematics teacher}
	{Education Community "2x2" \\
	\null\hfill \href{https://mathbaby.ru/}{mathbaby.ru}		
	 }
	{
		I tell additional chapters of mathematics in the school, such as linear algebra, group theory and probability.
	}

\end{entrylist}

%----------------------------------------------------------------------------------------
%	EDUCATION
%----------------------------------------------------------------------------------------

\cvsect{Education}

\begin{entrylist}
	\entry
		{2020--2022}
		{Master's Degree}
		{The Rhenish Friedrich Wilhelm University of Bonn \\
		\null\hfill \href{https://www.mathematics.uni-bonn.de/}{mathematics.uni-bonn.de}}
		{My main field is algebraic topology and I am currently working on my master's thesis <<Formality and coformality in rational homotopy theory>>. In addition I'm interested in algebra and logic, especially in homotopy type theory.}
	\entry
		{2015--2019}
		{Bachelor's Degree}
		{Lomonosov Moscow State University \\ \null\hfill Faculty of Computational Mathematics and Cybernetics.}
		{
			At the department of general mathematics, I was engaged in topological and metric spaces researches. My bachelor thesis is <<Fixed points and coincidences of ultrametric and some quasi-isometric spaces mappings>>.
		}
\end{entrylist}

%----------------------------------------------------------------------------------------
%	ADDITIONAL INFORMATION
%----------------------------------------------------------------------------------------

\begin{minipage}[t]{0.168\textwidth}
	\vspace{-\baselineskip} % Required for vertically aligning minipages

	\cvsect{Languages}
	
	\textbf{Russian} - native\\
	\textbf{English} - advanced\\
\end{minipage}
\hfill
\begin{minipage}[t]{0.2\textwidth}
	\vspace{-\baselineskip} % Required for vertically aligning minipages
	
	\cvsect{Skills}
	
	C++ 17, Haskell, Python,\\
	Git, Latex
\end{minipage}
\hfill
\begin{minipage}[t]{0.6\textwidth}
	\vspace{-\baselineskip} % Required for vertically aligning minipages
	
	\cvsect{Hobbies}
	
	My main hobbies are programming, teaching and music. As a programmer, I like sometimes to write small projects (on C++ or Haskell), related to math or gaming. As a teacher, I enjoy finding a way to tell people complicated things (like topology or algebra) in simple way. Finally, I am a musician and composer, I play piano, synths, and electric guitar. For several years I played in hard/progressive rock band.
	

	

\end{minipage}
\hfill


%----------------------------------------------------------------------------------------

\end{document}
